\documentclass[a4paper,12pt]{article}

\usepackage[utf8]{inputenc}
\usepackage{fullpage,url}
\usepackage{algorithm}
\usepackage{algorithmic}

\title{A Formal Proof of Bellman-Ford Algorithm}
\author{Yuto Takei \\ The University of Tokyo }

\begin{document}

\maketitle

\begin{abstract}
  This is the abstract.
\end{abstract}

\section{Introduction}

I want to \emph{emphasize} a word.

\section{Bellman-Ford Algorithm}

The Bellman-Ford algorithm is one of the algorithms to solve the single-source shortest paths problem, i.e., to give the shortest paths to all vertices from one given source vertex. To solve such a problem, Dijkstra's algorithm is widely known as an effective solution, yet it assums arcs to have non-negative length. The Bellman-Ford algorithm relaxed such the assumption and is able to compute the solution for the directed graphs with arcs with negative length present.

The single-source shortest paths problem is described as \( (G,s,l) \), where \( G \) represents a directed graph \( G=(V,E) \), \( s \in V \) is a source vertex, and \( l : E \rightarrow \mathbf{R} \) gives the length for all edges. The goal for this problem is either to produce the shortest path tree from \( s \) or to prove that there is a cycle, whose length is negative.


[TODO: following quoted from chapter 25.3 (p.532)]

\begin{algorithm}
\caption{Initialize-Single-Source($G$,$s$)}
\begin{algorithmic}
\FORALL{ vertex $ v \in V[G] $}
\STATE{$ d[v] \leftarrow \infty$}
\STATE{$ \pi[v] \leftarrow nil$}
\ENDFOR
\STATE{$d[s]\leftarrow 0$}
\end{algorithmic}
\end{algorithm}

\begin{algorithm}
\caption{Relax($u$,$v$,$l$)}
\begin{algorithmic}
\IF{$d[v]>d[u]+l(u,v)$}
\STATE{$d[v] \leftarrow d[u]+w(u,v)$}
\STATE{$\pi[v] \leftarrow u$}
\ENDIF
\end{algorithmic}
\end{algorithm}

\begin{algorithm}
\caption{BellmanFord($G$,$s$,$l$)}
\begin{algorithmic}
\STATE{InitializeSingleSource($G$,$s$)}
\FOR{$i = 1 \to |V[G]|-1$}
\FORALL{ edge $(u,v)\in E[G]$}
\STATE{Relax($u$,$v$,$s$)}
\ENDFOR
\ENDFOR
\FORALL{edge $(u,v)\in E[G]$}
\IF{$d[v]>d[u]+l(u,v)$}
\RETURN{False}
\ENDIF
\ENDFOR
\RETURN{True}
\end{algorithmic}
\end{algorithm}


\section{Quick Overview of Why3}

[TODO: quoting from ``Why3: Shepherd Your Herd of Provers'']

\section{Formal Proof}

\subsection{Specification}

\subsection{Proof}

\section{Conclusion}

\end{document}
