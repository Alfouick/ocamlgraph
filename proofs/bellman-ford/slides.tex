\documentclass{beamer}

\usetheme{Marburg}
\usecolortheme{beaver}
\usefonttheme{structurebold}

\usepackage{listings}
\usepackage{algorithm}
\usepackage{algorithmic}
\usepackage{listings}
\usepackage{amssymb,amsmath,amsthm}

\setbeamertemplate{footline}{
\hfill\scriptsize{\color{white}\insertframenumber}\hspace*{24pt}\vspace{8pt}}
\setbeamertemplate{navigation symbols}{}

\lstset{basicstyle=\footnotesize, columns=fullflexible}
\lstdefinelanguage{Ocaml}
{
basicstyle=\ttfamily,
keywordstyle=\color{blue},
morekeywords=[1]{exception,of,val},%
%keywordstyle=[1]{\color{red}},%
morekeywords=[2]{true,false},%
%keywordstyle=[2]{\color{blue}},%
otherkeywords={},%
commentstyle=\itshape,%
columns=[l]fullflexible,%
sensitive=true,%
morecomment=[s]{(*}{*)},%
escapeinside={*?}{?*},%
keepspaces=true,
literate=%
{<}{$<$}{1}%
{>}{$>$}{1}%
{<=}{$\le$}{1}%
{>=}{$\ge$}{1}%
{<>}{$\ne$}{1}%
{->}{$\rightarrow$}{2}%
{<->}{$\leftrightarrow$}{2}%
}

\lstdefinelanguage{Why3}
{
basicstyle=\ttfamily,
keywordstyle=\color{blue},
morekeywords=[1]{inductive,predicate,function,goal,type,use,import,theory,end,in,axiom,lemma,export,forall,constant,module,let,exception,match,with,exists,val,unit},%
%keywordstyle=[1]{\color{red}},%
morekeywords=[2]{true,false},%
%keywordstyle=[2]{\color{blue}},%
otherkeywords={},%
commentstyle=\itshape,%
columns=[l]fullflexible,%
sensitive=true,%
morecomment=[s]{(*}{*)},%
escapeinside={*?}{?*},%
keepspaces=true,
literate=%
{v1}{v$_1$}{1}%
{v2}{v$_2$}{1}%
{v3}{v$_3$}{1}%
{<}{$<$}{1}%
{>}{$>$}{1}%
{<=}{$\le$}{1}%
{>=}{$\ge$}{1}%
{<>}{$\ne$}{1}%
{/\\}{$\land$}{1}%
{\\/}{ $\lor$ }{3}%
{\ or(}{ $\lor$(}{3}%
{not\ }{$\lnot$ }{1}%
{not(}{$\lnot$(}{1}%
{->}{$\rightarrow$}{2}%
{<->}{$\leftrightarrow$}{2}%
}

\lstnewenvironment{why3}
{\lstset{language=Why3}}
{}

\begin{document}

\title{A Formal Proof of Bellman-Ford Algorithm}
\author[Yuto Takei]{Yuto Takei}
\institute{The University of Tokyo}

\maketitle

\begin{frame}{Table of contents}
\tableofcontents
\end{frame}

\section{Bellman-Ford Algorithm}
\frame{\sectionpage}

\subsection{Introduction}

\begin{frame}{Bellman-Ford Algorithm}

Solution for single-source shortest path problem $(G,s,l)$

\begin{itemize}
\item Directed graph $G(V,E)$
\item Source vertex $s\in V$
\item Length function $l: E\to \mathbf{K}$
\begin{itemize}
\item $\mathbf{K}$: Any monoid with total order (e.g. 
$\mathbf{N}, \mathbf{Z}, \mathbf{R}$).
\end{itemize}
\end{itemize}

\pause

The algorithm will either

\begin{itemize}
\item give shortest-paths to all vertices from $s$, or
\item assert the existence of a negative cycle.
\end{itemize}

\end{frame}

\begin{frame}{For example...}

\only<1>{\includegraphics[width=1.0\textwidth]{intro1.pdf}
}\only<2>{\includegraphics[width=1.0\textwidth]{intro2.pdf}
}\only<3>{\includegraphics[width=1.0\textwidth]{intro3.pdf}
}\only<4>{\includegraphics[width=1.0\textwidth]{intro4.pdf}
}

\begin{itemize} 
\only<1-2>{\onslide<2>{\item Gave distances from source to all vertices}}
\only<3-4>{\onslide<4>{\item Found a negative cycle}}
\end{itemize}

\end{frame}

\begin{frame}{Goal (revisited)}

Each node has a label $(\pi,d)$ containing parent and distance.

\includegraphics[width=1.0\textwidth]{intro5.pdf}

After the algorithm termination,
\begin{itemize}
\item all nodes have the smallest possible distance from $s$, and
\item the shortest path tree is built by traversing $\pi$
\end{itemize}

\end{frame}

\subsection{Mechanism}

\begin{frame}{How does it work?}

\begin{enumerate}

\item \alert{Initialization}

Set $(nil,\infty)$ for all vertex except $s$ having $(nil,0)$.\newline

\item \alert{Relaxing edges}

If $v$ is reachable with less distance by $(u,v)\in E$, update the
label. Perform this over all edges.

\only<1-2>{\begin{center}\onslide<2>\includegraphics[width=0.7\textwidth]{rlx1.pdf}\end{center}
}\only<3->{\begin{center}\includegraphics[width=0.7\textwidth]{rlx2.pdf}\end{center}
}\onslide<4>{Iterate for $|V|-1$ times. (You can stop when stable)}

\end{enumerate}

\end{frame}

\begin{frame}{Example}
\only<1>{Pass \#0 - just after initialization
\includegraphics[width=0.9\textwidth]{ex1.pdf}
}\only<2>{Pass \#1
\includegraphics[width=0.9\textwidth]{ex2.pdf}
}\only<3>{Pass \#1
\includegraphics[width=0.9\textwidth]{ex3.pdf}
}\only<4>{Pass \#1
\includegraphics[width=0.9\textwidth]{ex4.pdf}
}\only<5>{Pass \#1
\includegraphics[width=0.9\textwidth]{ex5.pdf}
}\only<6>{Pass \#2
\includegraphics[width=0.9\textwidth]{ex6.pdf}
}\only<7>{Pass \#3
\includegraphics[width=0.9\textwidth]{ex7.pdf}
}

\end{frame}

\begin{frame}{How does it work? (cont.)}

\begin{enumerate}
\setcounter{enumi}{2}

\item \alert{Still be able to relax?}

After $(|V|-1)$-th pass, all vertices should be exact.
Otherwise, a negative cycle exists.

\begin{center}
\only<1>{\includegraphics[height=0.6\textheight]{wst1.pdf}
}\only<2>{\includegraphics[height=0.6\textheight]{wst2.pdf}
}
\end{center}

\end{enumerate}

\end{frame}

\begin{frame}{$\mbox{\sc Bellman-Ford}(G,s,l)$}

\begin{algorithmic}[1]
\STATE{$\mbox{\sc Initialize-Single-Source}(G,s)$}
\FOR{$i=1\to |V[G]|-1$}
\FORALL{edge $(u,v) \in E[G]$}
\STATE{$\mbox{\sc Relax}(u,v,l)$}
\ENDFOR
\ENDFOR
\FORALL{edge $(u,v)\in E[G]$}
\IF{$d[v]>d[u]+l(u,v)$}
\RETURN{False}
\ENDIF
\ENDFOR
\RETURN{True}
\end{algorithmic}

(Cormen et al. Introduction to Algorithms, p 532)

\end {frame}

\section{Implementation in OCamlgraph}
\frame{\sectionpage}

\begin{frame}[fragile]{Work that has been done}

\begin{enumerate}
\item Implementation of Bellman-Ford algorithm

\begin{lstlisting}[language=Ocaml]
exception Negative_cycle of G.E.t list

val all_shortest_paths:
             G.t -> G.V.t -> W.t H.t

val find_negative_cycle_from:
             G.t -> G.V.t -> G.E.t list
\end{lstlisting} \vspace*{10pt}

\pause
\item Negative cycle detection from any source

\begin{lstlisting}[language=Ocaml]
val find_negative_cycle: G.t -> G.E.t list
\end{lstlisting}

\end{enumerate}

\end{frame}


\begin{frame}{Negative cycle detection from any source}

\vspace*{8pt}
\only<1>{Want to ensure that $G$ is free of any negative cycles.\vspace*{-18pt}
}\only<2>{Checking from specific $s$ is not exhaustive.\vspace*{-14pt}
}\only<3>{\alert{Solution:} To give a ``virtual'' $s$ that can reach all nodes.}

\begin{center}
\only<1>{\includegraphics[height=0.8\textheight]{s1.pdf}
}\only<2>{\includegraphics[height=0.8\textheight]{s2.pdf}
}\only<3>{\includegraphics[width=0.8\textwidth]{s3.pdf}
}\end{center}

\only<3>{$s$ connects to one of the elements in every top ancestor among
strongly connected components.\vspace{100pt}}

\end{frame}


\begin{frame}[fragile]{Work that has been done (cont.)}

\vspace*{16pt}\begin{enumerate}
\setcounter{enumi}{2}
\item Incremental graph builder w/o negative cycle\newline

\only<1>{
Keep track of
\begin{itemize}
\item the set of ancestors $S$, and
\item distances from each ancestor to every node $d_s(v)$
\end{itemize}

\begin{center}
\includegraphics[width=0.9\textwidth]{inc1.pdf}
\end{center}
}

\only<2-3>
{
When adding
\begin{itemize}
\item\alert{vertex $v$}, just add $v$ to $S$.
\item\alert{edge $(u,v)$}, if $v\in S$, check if $(u,v)$ causes a loop.\newline
If not, $S\leftarrow S\backslash v$. Propagate distance to descendants.
\end{itemize}

\begin{center}
\only<2>{\includegraphics[width=0.8\textwidth]{inc2.pdf}}
\only<3>{\includegraphics[width=0.8\textwidth]{inc3.pdf}}
\end{center}
}

\only<4-5>
{
When removing
\begin{itemize}
\item\alert{vertex $v$}, remove all edges $(u,v)$, $(v,w)$ then $v$ itself.
\item\alert{edge $(u,v)$}, correct the distances of $v$ and propagate.
\end{itemize}

\begin{center}
\only<4>{\includegraphics[width=0.65\textwidth]{inc4.pdf}}
\only<5>{\includegraphics[width=0.65\textwidth]{inc5.pdf}}
\end{center}
}
\end{enumerate} \vspace*{100pt}

\end{frame}

\section{Formal proof with Why3}
\frame{\sectionpage}

\subsection{Specification}

\begin{frame}[fragile]{Specification}

\begin{itemize}
\item \alert{Graph}

\begin{lstlisting}[language=why3]
type vertex
constant vertices : set vertex
constant edges: set (vertex, vertex)
constant s : vertex
function weight vertex vertex : int
\end{lstlisting}

\item \alert{Algorithm interface}

\begin{lstlisting}[language=why3]
type dist = Finite int | Infinite
exception Negative_cycle
val bellman_ford : unit -> map vertex dist
\end{lstlisting}
\end{itemize}

\end{frame}

\begin{frame}[fragile]{Specification}

\begin{itemize}
\item \alert{Path} as an inductive predicate
\small
\begin{lstlisting}[language=why3]
inductive path (v1 v2: vertex) (n: int) =
| path_empty:
    forall v: vertex. path v v 0
| path_succ:
    forall v1 v2: vertex, n: int. path v1 v2 n ->
    forall v3: vertex. mem (v2, v3) edges ->
    path v1 v3 (n + (weight v2 v3))
\end{lstlisting}
\normalsize

\item \alert{Shortest path} and \alert{non-reachability}
\small
\begin{lstlisting}[language=why3]
predicate shortest_path (v1 v2: vertex) (n: int) =
  (path v1 v2 n) /\
  (forall m: int. m < n -> not (path v1 v2 m))

predicate no_path (v1 v2: vertex) =
  forall n: int. not (path v1 v2 n)
\end{lstlisting}
\normalsize
\end{itemize}
\end{frame}

\begin{frame}[fragile]{Specification}

\begin{itemize}

\item \alert{Precondition} and \alert{postcondition}

\begin{lstlisting}[language=why3]
let bellman_ford () =
  { (* nothing *) }
  ...
  { forall v: vertex. mem v vertices ->
    match result[v] with
    | Infinite -> no_path s v
    | Finite n -> shortest_path s v n
    end }
  | Negative_cycle ->
  { exists v: vertex. mem v vertices /\
    exists n: int. n < 0 /\ path v v n }
\end{lstlisting}

\end{itemize}

\end{frame}

\subsection{Proof}

\begin{frame}[fragile]{Termination}

Loop variants are trivial and easily given.\vspace*{12pt}

\begin{algorithmic}[1]
\STATE{$\mbox{\sc Initialize-Single-Source}(G,s)$}
\FOR{\alert{$i=1\to |V[G]|-1$}}
\FORALL{\alert{edge $(u,v)\in E[G]$}}
\STATE{$\mbox{\sc Relax}(u,v,l)$}
\ENDFOR
\ENDFOR
\FORALL{\alert{edge $(u,v)\in E[G]$}}
\IF{$d[v]>d[u]+l(u,v)$}
\RETURN{False}
\ENDIF
\ENDFOR
\RETURN{True}
\end{algorithmic}

\end{frame}

\begin{frame}{Strategy for correctness}

Invariant during the algorithm execution for example:

\begin{itemize}

\item If $v$ has a \alert{finite} distance $n$

$\to$ Path from $s$ to $v$ with distance $n$ exists

\item If $v$ has an \alert{infinite} distance

$\to$ $v$ is not yet reachable at given point

\end{itemize}

\pause

\vspace{24pt}Need to keep this condition through
\begin{itemize}
\item every pass
\item every edge relaxation during one pass
\end{itemize} 

\end{frame}

\begin{frame}{Strategy for correctness}

\vspace*{16pt}

\only<1>{After $i$-th pass\vspace*{-16pt}
\begin{center}
\includegraphics[height=0.8\textheight]{inv1.pdf}
\end{center}
}\only<2>{Executing pass \#$i+1$, relaxed some edges\vspace*{-16pt}
\begin{center}
\includegraphics[height=0.8\textheight]{inv2.pdf}
\end{center}
}

\end{frame}

\begin{frame}[fragile]{Invariant at a glance}
\footnotesize
\begin{lstlisting}[language=why3]
predicate invar (m: distmap) (pass: int) (via: set edge) =
  forall v: vertex. mem v vertices ->
  match m[v] with
  | Finite n ->
    negcycle \/
    (not negcycle /\ (exists n d: int. shortest_path s v n d) /\
    forall n' d': int. shortest_path s v n' d' ->
      (0 <= d' < pass -> n = n') /\
      (d' = pass ->
        (exists u: vertex. mem (u, v) edges /\
          shortest_path s u (n' - weight u v) (pass - 1)) /\
        (forall u: vertex. mem (u, v) edges /\
          shortest_path s u (n' - weight u v) (pass - 1) ->
            mem (u, v) via -> n = n')))
  | Infinite ->
    (forall d: int. 0 <= d < pass ->
      forall n: int. not (path s v n d)) /\
    (forall u: vertex. mem (u, v) via ->
      forall d: int. 0 <= d < pass ->
        forall n: int. not (path s u n d))
  end
\end{lstlisting}
\end{frame}


\begin{frame}[fragile]{Problems to solve}

\begin{itemize}

\item Give an appropriate property for negative cycle

\item Guarantee an existence of ``simple path''.

\begin{lstlisting}[language=why3]
lemma reach_less_than_n:
  forall v1 v2: vertex. mem v1 vertices ->
  forall d n: int. path v1 v2 n d ->
  exists n' d': int.
    d' < cardinal vertices /\
    path v1 v2 n' d'
\end{lstlisting}

\item Reduce hand-proving (e.g. Coq) as much as possible

\end{itemize}
\end{frame}

\begin{frame}

\Huge \alert{Demo}\vspace{8pt}

\large Proving infinite path properties on Why3

\end{frame}

\section{Conclusion}

\frame{\sectionpage}
\begin{frame}{Conclusion and Perspectives}
Done:

\begin{itemize}
\item Implemented Bellman-Ford algorithm in Ocamlgraph
\begin{itemize}
\item Normal traversal
\item Exhaustive negative cycle detection
\item Incremental graph builder
\end{itemize}

\item Proved some characteristics of the algorithm
\begin{itemize}
\item Termination (including safety)
\item Properties of unreachable nodes
\end{itemize}
\end{itemize}

Do:
\begin{itemize}
\item Finish the formal proof
\item Compile into a paper
\end{itemize}
\end{frame}

\end{document}
