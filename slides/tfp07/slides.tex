\documentclass{beamer}

\usepackage{pgf}
\usepackage{calc}
\usepackage{amsmath}
\usepackage{amssymb}
\usepackage{amsthm}
\usepackage[latin1]{inputenc}
\usepackage{beamerthemesplit}

\usetheme{Cea}

\title{Designing a Generic Graph Library using ML Functors}
\author{S. Conchon \and J.-C. Filli\^atre \and J. Signoles}
\date{April 4$^{\mbox{th}}$, 2007}

\begin{document}

\setbeamertemplate{navigation symbols}{}

\newtheorem{prop}{Property}

\frame{\titlepage}

\section[Outline]{}
\frame{\tableofcontents}

\section{Goal}


\frame{
  \frametitle{Facts}
  \begin{itemize}
  \item SystemC has a top level loop which simulates events.
  \item A loss of accuracy on the top level loop of a simulation
yields immediately a bad result.
  \end{itemize}
}

\frame{
  \frametitle{What we want}
\begin{center}
\fbox{Put graphic in here!}
\end{center}
}

\begin{frame}[containsverbatim]
  \frametitle{Simple example of goal}
\begin{minipage}{5cm}
\begin{verbatim}
int main(int i) {
  int j=0;
  j=2*i;
  if (2*i==j) {
    return 314;
  }
  return 2;
}
\end{verbatim}
\end{minipage}
\begin{minipage}{5cm}

Represention of the return value:
\begin{center}
\fbox{Put graphic in here!}
\end{center}
\end{minipage}
\end{frame}

\begin{frame}[containsverbatim]
  \frametitle{Example of a property}
\begin{prop}
First assert in {\tt checkTwoDigitAdder} is always true.
\end{prop}
\begin{proof}
According to data hypergraph, formula to prove is:
$$(a+b)\& 1 = 0 \mathtt{xor} ( (a\& 1) \mathtt{xor} (b\& 1))$$
which becomes:
$$(a_1+a_2<<1 + b_1+b_2<<1)\& 1 =  (a\& 1) \mathtt{xor} (b\& 1)$$
$$(a_1+ b_1)\& 1 =  a_1 \mathtt{xor} b_1$$
\end{proof}
\end{frame}

\end{document}
    
